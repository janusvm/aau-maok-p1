% Original source and additional resources is available on:
% https://github.com/janusvm/aau-maok-p1

\documentclass[12pt,a4paper,oneside,final]{article}

% PREAMBLE ---------------------------------------------------------------------
\title{
  Optimering af gaslager\\
  \large P1-projektforslag \\
  Matematik-{\O}konomi
}
\author{
  Janus Valberg-Madsen\\
  \href{mailto:janus@math.aau.dk}{\texttt{<janus@math.aau.dk>}}
}
\date{14.\ oktober 2019}

% Packages
\usepackage[T1]{fontenc}
\usepackage[utf8]{inputenc}
\usepackage[danish]{babel}
\usepackage[danish]{translator}
\usepackage{hyperref}
\usepackage[charter]{mathdesign}
\usepackage{inconsolata}
\usepackage[labelfont=bf,font=it]{caption}
\usepackage[margin=2.5cm]{geometry}
\usepackage[shortlabels]{enumitem}
\usepackage[dvipsnames]{xcolor}
\usepackage{graphicx}
\usepackage{tikz}
\usepackage{mathtools}
\usepackage{bbm}
\usepackage{bm}


% Commands
\newcommand{\N}{\mathbb{N}}
\newcommand{\Z}{\mathbb{Z}}
\newcommand{\R}{\mathbb{R}}
\newcommand{\1}{\mathbbm{1}}
\newcommand{\e}{\mathrm{e}}
\newcommand{\vzero}{\bm{0}}
\newcommand{\vx}{\bm{x}}
\newcommand{\vb}{\bm{b}}
\newcommand{\vc}{\bm{c}}
\newcommand{\vs}{\bm{s}}
\newcommand{\nodemark}[1]{\tikz[baseline=-0.5ex]\node[#1] at (0,0) {};}


% Settings
\setlist{itemsep=0pt}
\usetikzlibrary{arrows.meta,backgrounds,calc,calendar,positioning}
\tikzstyle{afternoon} = [shape=rectangle,minimum size=1.5em,fill=SkyBlue]
\tikzstyle{morning}   = [shape=rectangle,minimum size=1.5em,fill=Lavender]
\tikzstyle{special}   = [shape=rectangle,minimum size=1.5em,fill=YellowGreen]
\tikzstyle{asym}      = [shape=rectangle,minimum size=1.0em,fill=SkyBlue]
\tikzstyle{msym}      = [shape=rectangle,minimum size=1.0em,fill=Lavender]
\tikzstyle{ssym}      = [shape=rectangle,minimum size=1.0em,fill=YellowGreen]
\tikzstyle{point}     = [text=white,fill,shape=circle,minimum size=5pt,inner sep=0pt]



% MAIN -------------------------------------------------------------------------
\begin{document}
\maketitle

\begin{center}
  \begin{tikzpicture}[scale=1.5]
  \node [point] (a) at (0,2) {};
  \node [point] (z) at (6,2) {};
  \foreach \t in {1,...,5} {
    \foreach \q in {0,...,4} {
      \node [point] (\t\q) at (\t,\q) {};
    }
  }
  \begin{scope}[on background layer]
    \foreach \q in {0,...,4} {
      \draw[black!30] (a) -- (1\q);
      \draw[black!30] (5\q) -- (z);
      \foreach \t [evaluate=\t as \nextt using int(\t+1)] in {1,...,4} {
        \foreach \nq [evaluate=\nq as \nextq using int(\q+\nq)] in {-2,-1,0,1,2} {
          \ifthenelse{\nextq > 4 \OR \nextq < 0}{}{
            \draw[black!30] (\t\q) -- (\nextt\nextq);
          }
        }
      }
    }
    \draw [red,line width=2pt] (a) -- (11) -- (23) -- (34) -- (42) -- (51) -- (z);
    \foreach \p in {a,11,23,34,42,51,z} \node [point,fill=red,minimum size=10pt] at (\p) {};
  \end{scope}
\end{tikzpicture}

\end{center}

\tableofcontents


\clearpage
\section{Basisproblem}
Forestil jer, at vi kontrollerer et gaslager---en fysisk enhed, hvori vi kan lagre naturgas---i en given periode på et år.
Med ``kontrollerer'' menes her, at vi ikke selv ejer enheden, men at vi har indgået en kontrakt med ejeren om at leje enheden.
Vi er interesserede i at maksimere profitten (eller minimere tabene), som opnås ved at købe og sælge gas i løbet af lejeperioden.

Antag hertil, at vi har en såkaldt \emph{forward}-kurve for prisen på gas, dvs.\ en række kontrakter med modparter på markedet, som giver os forudbestemte priser på gassen til hver tidspunkt i lejeperioden.
Til hvert tidspunkt i denne periode kan vi købe eller sælge en begrænset mængde gas, og der er øvre og nedre grænser på størrelsen af gaslageret (enten fysiske eller kontraktbestemte).
Når lejeperioden udløber, køber ejeren af lageret den efterladte mængde gas af os.
Vi ser desuden bort fra de omkostninger, der måtte være i en sådan lejekontrakt.

\subsection{Matematisk formulering}
Lad \(q_{t}\) angive mængden af gas (målt i \(\epsilon\times\)MWh, hvor \(\epsilon\) er størrelsen af en ``enhed'' gas) lagret til tid \(t\) (hvor \(t = 0\) er i dag, \(t = 1\) er næste tidsskridt osv.\ indtil slutningen af lejeperioden, \(T\)) og lad \(\Delta q_{t} = q_{t} - q_{t-1}\) angive ændringen i lageret fra den ene dag til den anden.
Givet parametrene
\begin{itemize}
\item \(q_{0}\), antal enheder af gas på lageret i starten af perioden
\item \(q_{\min}\) og \(q_{\max}\), den nedre og øvre grænse på mængden, der kan være på lageret
\item \(u_{\max}, i_{\max} \in \N\), antal enheder, der hhv.\ kan tages ud og sættes ind mellem to tidsperioder
\item \(p_{t}\), en forward-pris i euro på en enhed gas, defineret for \(t \in \{1, \dotsc, T\}\)
\item \(r\), en fast, årlig diskonteringsrente
\end{itemize}
maksimér nutidsværdien af profitten mht.\ \(\Delta q_{t}\), dvs.\
\begin{equation}
  \label{eq:1}
  \max_{\Delta q_{t}}
  \left\{
    - \sum_{t = 1}^{T} \e^{-r\tfrac{t}{T}} \Delta q_{t} p_{t} + \e^{-r} q_{T} p_{T}
  \right\} ,
\end{equation}
under bibetingelserne
\begin{enumerate}
\item \(q_{\min} \leq q_{t} \leq q_{\max}\)
\item \(\Delta q_{t} \in
    \left\{
    -u_{\max}, -u_{\max}+1, \dotsc, -1, 0, 1, \dotsc, i_{\max} - 1, i_{\max}
    \right\}\)
\end{enumerate}
for \(t \in \{1, \dotsc, T\}\), hvor \(q_{t} = q_{0} + \sum_{s = 1}^{t} \Delta q_{s}\) som nævnt ovenfor er mængden lagret til tid \(t\).

\clearpage
\subsection{Løsning af problemet med grafteori}
Eftersom antallet af tidsperioder og antallet af mulige mængder gas på lageret begge er diskrete, dvs.\ \(t \in \{1, \dotsc, T\}\) og \(q_{t} \in \{q_{\min}, q_{\min}+1, \dotsc, q_{\max}\}\), kan problemet repræsenteres som en \emph{orienteret, vægtet graf}, som på skitsen nedenfor på Figur~\ref{fig:graph}:

\begin{figure}[htbp]
  \centering
  \begin{tikzpicture}[scale=1.5,-{Latex[angle'=60,length=4pt]},shorten >=1pt]
  \foreach \t [evaluate=\t as \myt using int(\t+1)] in {1, 2, ..., 5} {
    \foreach \q in {0, 1, ..., 3} {
      \node[point] at (1.25*\t,1.25*\q) (p\t\q) {};
    }
  }
  \foreach \t [evaluate=\t as \nextt using int(\t+1)] in {1, 2, ..., 4} {
    \foreach \q in {0, 1, ..., 3} {
      \foreach \nq [evaluate=\nq as \nextq using int(\q+\nq)] in {-1, 0, 1} {
        \ifthenelse{\nextq > 3 \OR \nextq < 0}{}{
          \draw (p\t\q) -- (p\nextt\nextq);
        }
      }
    }
  }
  \node[point] at (0,1.25) (q0) {};
  \foreach \q in {0, 1, 2} {
    \draw (q0) -- (p1\q);
  }
  \begin{scope}[color=red,very thick,-{Latex[angle'=65,length=6pt]},shorten >=0pt]
    \draw (q0)  -- (p12);
    \draw (p12) -- (p22);
    \draw (p22) -- (p33);
    \draw (p33) -- (p42);
    \draw (p42) -- (p51);
  \end{scope}
  \node at (q0) (lq0) [label=left:\(q_{0}\)] {};
  \node at (p10) (qmin) [label=left:\(q_{\min}\)] {};
  \node at (p13) (qmax) [label=left:\(q_{\max}\)] {};
  \node (t1) [below of=p10,yshift=1em] {\(t=1\)};
  \node (tT) [below of=p50,yshift=1em] {\(t=T\)};
  \node at (p12) (q1) [label=above:\(q_{1}\)] {};
  \node at (p22) (q2) [label=above:\(q_{2}\)] {};
  \node at (p33) (q3) [label=below:\(q_{3}\)] {};
  \node at (p42) (q4) [label=below:\(q_{4}\)] {};
  \node at (p51) (q5) [label=below:\(q_{5}\)] {};
\end{tikzpicture}

  \captionsetup{width=0.75\textwidth}
  \caption{
    \small
    Skitse af problemgrafen (kantvægte er udeladt for læsbarhed), med \(u_{\max} = i_{\max} = 1\).
    Et eksempel på en vej gennem grafen er indtegnet med rødt.
    Opadgående pile betegner indkøb af gas, mens nedadgående piler betegner salg.
  }
  \label{fig:graph}
\end{figure}

Vælges kantvægtene til at repræsentere udgifter og indtjening på hhv. køb og salg af gas, så er dette et \emph{længste vej}-problem, og det kan løses med algoritmer fra grafteori.
Hertil findes algoritmer, der kan finde den længste (eller korteste) vej igennem en graf af generel struktur, men I opfordres her til at tænke over, hvordan strukturen i dette specifikke problem kan udnyttes til at designe en mere specialiseret algoritme.
Som en del af projektet skal I således implementere en sådan algoritme i et programmeringssprog som Julia, Python eller lignende.

I genkender måske formen af dette problem som et \emph{heltalsprogrammerings}-problem (eng.: \emph{integer linear programming} (ILP)), dvs.\ et optimeringsproblem, som kan opskrives på standardformen
\begin{alignat*}{2}
 & \text{maksimér}            &\quad  & \vc^{\top} \vx  \\
 & \text{under bibetingelser} &\quad  & A \vx + \vs = \vb , \\
 &                            &\quad  & \vs \geq \vzero , \\
 &                            &\quad  & \vx \geq \vzero , \\
 & \text{og}                  &\quad  & \vx \in \Z^{n} ,
\end{alignat*}
hvor \(\vc,\vb\) og \(A\) er givne parametre, og hvor \(\vx\) og \(\vs\) er variable.
Dermed kan det ganske vist løses med software-pakker som GLPK, Excel Solver Add-in, osv., men I anbefales kun at bruge sådanne programmer til at tjekke resultaterne fra jeres egen algoritme.
En af udvidelserne i næste afsnit gør desuden objektfunktionen ikke-lineær, og så kan disse metoder ikke længere anvendes.


\clearpage
\section{Udvidelser til problemet}
Når I har løst basisproblemet, kan I arbejde videre med nedenstående udvidelser.
Overvej gerne hertil, hvordan I gør jeres implementation af løsningsalgoritmen generel nok til at løse problemer af alle typer (basisproblem, med én udvidelse, med begge udvidelser).

\subsection{Tidsafhængige begrænsninger}
Grænser på lageret vil i mere realistiske modeller for dette problem ikke være faste henover et år;
for eksempel vil \(q_{\min}\) typisk være højere om vinteren for at kunne imødekomme den øgede efterspørgsel på strøm, hvis produktionen er lav en dag pga.\ fejl på et produktionsanlæg.
Derfor er den første udvidelse til problemet at løse det med tidsafhængige begrænsninger, dvs.\
\begin{enumerate}
\item \(q_{\min,t} \leq q_{t} \leq q_{\max,t}\)
\item \(\Delta q_{t} \in
    \left\{
    -u_{\max,t} , -u_{\max,t}+1 , \dotsc, -1, 0, 1, \dotsc, i_{\max,t} - 1 , i_{\max,t}
    \right\}\)
\end{enumerate}
for alle \(t\).
Med andre ord, så afhænger \(q_{\min}, q_{\max}, u_{\max}, i_{\max}\) fra basisproblemet nu af \(t\).

\subsection{Slutvolumen og straffaktor}
Som en del af lejekontrakten i problemet kan det tænkes, at vi forpligter os til at aflevere lageret med en aftalt mængde gas, \(q_{\mathrm{goal}}\).
Hvis ikke vi rammer denne værdi præcist, bliver prisen, som modparten køber den lagrede gas for, reduceret med en \emph{straffaktor}, og vi får dermed ikke den fulde pris for gassen.
I denne udvidelse er dermed givet to nye parametre,
\begin{itemize}
\item \(\alpha\), en straffaktor for prisen (\(0 < \alpha < 1\))
\item \(q_{\mathrm{goal}}\), den aftalte mængde gas lagret til tid \(T\)
\end{itemize}
og maksimeringsproblemet bliver således
\begin{equation}
  \max_{\Delta q_{t}}
  \left\{
    - \sum_{t = 1}^{T} \e^{-r\tfrac{t}{T}} \Delta q_{t} p_{t} + \e^{-r} q_{T}p_{T}
    \left(
      1 - \alpha \cdot \1 [q_{T} \neq q_{\mathrm{goal}}]
    \right)
  \right\} ,
\end{equation}
hvor \(\1[\cdot]\) er en indikatorfunktion, dvs.\ den er lig med 1, når udsagnet (her, at \(q_{T} \neq q_{\mathrm{goal}}\)) er sandt, og 0 ellers.
Dermed er værdien af udtrykket i parentesen 1, hvis vi rammer \(q_{\mathrm{goal}}\) præcist, og \(1 - \alpha\), hvis vi ikke gør.

I denne sammenhæng kunne det være relevant at undersøge, om der findes tilfælde, hvor det kan betale sig for os \emph{ikke} at ramme \(q_{\mathrm{goal}}\), og hvilken effekt størrelsen af \(\alpha\) har.


\clearpage
\section{Data til test af algoritme}

\subsection{Basisproblem}
Til at starte med kan I tage udgangspunkt i følgende eksempler på parametre for basisproblemet:
\begin{align*}
  q_{\min} &= 0 & q_{0} &= 5 \\
  q_{\max} &= 10 & r &= 0.04 \\
  u_{\max} &= 4 &  T &= 12 \\
  i_{\max} &= 4 &  (p_{t})_{t=1}^{T} &= (20, 22, 25, 18, 15, 15, 20, 19, 21, 12, 22, 25)
\end{align*}
Dette svarer til en lejeperiode på et år med priser givet for hver måned.

\subsection{Udvidelser}
Med udgangspunkt i samme \(q_{0}, r, T\) og priser som i basisproblemet, er følgende eksempel for parametre for udvidelserne givet:
\begin{align*}
  q_{\mathrm{goal}} & = 5   & (q_{\min,t})_{t=1}^{T} & = (0, 0, 0, 4, 4, 6, 6, 4, 4, 0, 0, 0)          \\
  \alpha            & = 0.7 & (q_{\max,t})_{t=1}^{T} & = (10, 10, 10, 10, 8, 8, 8, 8, 10, 10, 10, 10)  \\
                    &       & (u_{\max,t})_{t=1}^{T} & = (4, 4, 2, 2, 1, 1, 1, 1, 2, 2, 4, 4)          \\
                    &       & (i_{\max,t})_{t=1}^{T} & = (4, 4, 2, 2, 1, 1, 1, 1, 2, 2, 4, 4)
\end{align*}


\section{Litteratur}
Som understøttende litteratur til projektet kan I bruge bogen:
\begin{center}
  Kenneth H.\ Rosen,
  \emph{Discrete Mathematics and Its Applitcations, Global Edition},\\
  McGraw-Hill Higher Education,
  7.\ udgave,
  ISBN: 978-0-07-131501-2
\end{center}
Bogens kapitel 10 og 11 omhandler grafer og træer og kan være et godt udgangspunkt for at danne jer et overblik og en forståelse af teorien.
Senere udgaver af bogen kan også bruges.

Bemærk dog, at bogens definitioner, sætninger og beviser er skrevet med læserens intuition in mente, og derved mangler der ofte  matematisk præcision.
Her er det jeres opgave at overveje, hvordan disse definitioner, sætninger, osv.\ kan formuleres mere præcise, f.eks.\ gennem notation.

\end{document}

%%% Local Variables:
%%% mode: latex
%%% TeX-master: t
%%% End:
