\section{Vejledermøder}
\label{sec:meetings}
\subsection{Tidspunkter}
\label{subsec:times}
Der er mulighed for at have op til \textbf{\'et vejledermøde \`a højst halvanden time om ugen}, og I kan vælge imellem de markederede dage i kalenderen nedenfor:
\begin{center}
  \begin{tikzpicture}[every calendar/.style={week list,month label above centered,day text={\%d=}}]
  \calendar (oct) [dates=2019-10-01 to 2019-10-last] at (0,0) if (weekend) [Gray];
  \calendar (nov) [dates=2019-11-01 to 2019-11-last] at (5.5,0) if (weekend) [Gray];
  \calendar (dec) [dates=2019-12-01 to 2019-12-last] at (11,0) if (weekend) [Gray];

  % Available dates
  \begin{scope}[on background layer]
    % October
    \foreach \d in {18,23,25,28} \node[afternoon] at (oct-2019-10-\d) {};
    \foreach \d in {21} \node[morning] at (oct-2019-10-\d) {};
    % November
    \foreach \d in {20,22,27,29} \node[afternoon] at (nov-2019-11-\d) {};
    \foreach \d in {04,11} \node[morning] at (nov-2019-11-\d) {};
    \foreach \d in {01,08,13} \node[special] at (nov-2019-11-\d) {};
    % December
    \foreach \d in {09,10} \node[afternoon] at (dec-2019-12-\d) {};
    \foreach \d in {05} \node[morning] at (dec-2019-12-\d) {};
    \foreach \d in {02,04} \node[special] at (dec-2019-12-\d) {};
  \end{scope}
\end{tikzpicture}

\end{center}
hvor
\begin{itemize}[label={},wide,labelindent=0pt,topsep=1ex]
\item \nodemark{msym} angiver dage, hvor jeg er tilgængelig \textbf{om formiddagen (9:00--12:00)}
\item \nodemark{asym} angiver dage, hvor jeg er tilgængelig \textbf{om eftermiddagen (12:30--15:30)}
\item \nodemark{ssym} angiver dage, hvor jeg er tilgængelig \textbf{kun hvis \emph{begge} mine grupper ønsker at have et møde den pågældende dag}.
  Dette skyldes at jeg disse dage ikke har andre ærinder på Basis, så jeg ville skulle tage derud udelukkende for at holde vejledermøder.
\end{itemize}

\clearpage
\subsection{Booking af møder}
\label{subsec:booking}
Det er op til jer at indkalde til vejledermøder, når I har brug for dem.
Mødeindkaldelser skal ske via kalenderinvitation på \url{https://mail.aau.dk}:
\begin{enumerate}[itemsep=0pt]
\item Opret et møde på en dag, hvor jeg er tilgængelig (ifølge Afsnit \ref{subsec:times})
\item Tilføj mig (\texttt{janus@math.aau.dk}) og alle gruppemedlemmer som mødedeltagere
\item Vælg et tidsinterval på op til 1½ time, sådan at det stadig er muligt for mig at holde et møde samme dag med den anden gruppe, hvis de ønsker det
\item Vedhæft arbejdsblade i PDF-format, og skriv læsevejledningen til dem ind i mødebeskrivelsen (se Afsnit \ref{sec:workingpapers} for en uddybelse af dette)
\item Skriv dagsordenen og spørgsmål ind i mødebeskrivelsen
\item Angiv jeres grupperum i lokale-feltet
\end{enumerate}
Indkaldelsen (med arbejdsblade og alle de nævnte informationer) skal være afsendt \textbf{senest kl.\ 16 to dage før den ønskede mødedag} for at sikre, at jeg har en chance for at nå at læse arbejdsbladene.
Hvis mødet ligger om mandagen, vil jeg dog kræve, at indkaldelsen foreligger allerede \textbf{fredag kl.\ 16}.

Ønsker I et vejledermøde på en \nodemark{ssym}-dag, skal møderne for begge grupper ligge i forlængelse af hinanden.
Sørg i alle tilfælde for at koordinere mødetider med den anden gruppe.

\subsection{Dagsorden}
\label{subsec:agenda}
Alle mødeindkaldelser skal indeholde en dagsorden for mødet.
I kan med fordel bruge følgende liste som en skabelon for, hvad en sådan dagsorden kunne indeholde:
\begin{enumerate}[itemsep=0pt]
\item Mødets formål (hvad mødes vi for?)
\item Status på gruppearbejdet (hvad har I hver især lavet siden sidst, hvordan fungerer gruppearbejdet, har I haft nogle problemer, overholder I tidsplanen, etc.)
\item Svar på spørgsmål (hvis I har angivet dem i mødeindkaldelsen)
\item Kommentarer til arbejdsbladene (hvis I har vedhæftet dem i mødeindkaldelsen)
\item Plan for næste uge (hvad skal I lave indtil næste gang vi ses?)
\item Eventuelt (andre ting, som I gerne vil have vendt med mig)
\item Evaluering af mødet (blev mødets formål opfyldt?)
\end{enumerate}
Der må selvfølgelig gerne være flere punkter, men jeg vil anbefale, at I altid som minimum har de ovenstående med, så I får så meget ud af møderne som muligt.

\subsection{Referat}
\label{subsec:summary}
\'En af gruppemedlemmerne skal tage referat af vejledermødet, som efter mødet sendes til hele gruppen og mig via email.
Undgå venligst at vedhæfte referatet som fil, men skriv istedet indholdet direkte ind i emailen (dette gør det bl.a.\ lettere at søge i referater senere).
