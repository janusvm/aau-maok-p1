\section{Hvad jeg kan hjælpe med}
\label{sec:help}
Foruden det faglige indhold i rapporten og det organisatoriske i arbejdsprocessen, kan jeg også hjælpe med tekniske ting omkring
\begin{itemize}
\item \LaTeX{} og dokument-/projektstruktur
\item Git og GitHub
\item Programmering i Julia, R eller Python
\end{itemize}
Jeg hjælper \emph{ikke} med: Overleaf, Microsoft Office, Dropbox/Google Drive/OneDrive, Maple, MATLAB, eller andre proprietære CAS-værktøjer.
I må selvfølgelig stadig godt bruge dem, blot I ikke forventer, at jeg kan svare på spørgsmål eller løse problemer, der involverer dem.

\subsection{Kommentarer via GitHub}
\label{subsec:github}
Hvis I har jeres projekt under versionsstyring på \href{https://github.com}{GitHub} kan I invitere mig som \emph{Collaborator} (brugernavn: \href{https://github.com/janusvm}{\texttt{janusvm}}).
Så vil jeg kunne kommentere direkte på kode, oprette/deltage i Issues, og have adgang til seneste version af jeres projekt.
Dette kan gøre det nemmere for mig at hjælpe jer, hvis I f.eks.\ har problemer med \LaTeX{} eller Julia/R/Python og har brug for hjælp til fejlfinding.

Brugen af Git og GitHub er ikke et krav, men det er en stor fordel for jer at kende til.
I det omgang, I skulle ønske det, har jeg mulighed for at hjælpe jer i gang med et GitHub-baseret workflow.
