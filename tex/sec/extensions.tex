\section{Udvidelser til problemet}
\label{sec:extensions}
Når I har løst basisproblemet, kan I arbejde videre med nedenstående udvidelser.
Overvej gerne hertil, hvordan I gør jeres implementation af løsningsalgoritmen generel nok til at løse problemer af alle typer (basisproblem, med én udvidelse, med begge udvidelser).

\subsection{Tidsafhængige begrænsninger}
\label{subsec:timdep}
Grænser på lageret vil i mere realistiske modeller for dette problem ikke være faste henover et år;
for eksempel vil \(q_{\min}\) typisk være højere om vinteren for at kunne imødekomme den øgede efterspørgsel på strøm, hvis produktionen er lav en dag pga.\ fejl på et produktionsanlæg.
Derfor er den første udvidelse til problemet at løse det med tidsafhængige begrænsninger, dvs.\
\begin{enumerate}
\item \(q_{\min,t} \leq q_{t} \leq q_{\max,t}\)
\item \(\Delta q_{t} \in
    \left\{
    -u_{\max,t} , -u_{\max,t}+1 , \dotsc, -1, 0, 1, \dotsc, i_{\max,t} - 1 , i_{\max,t}
    \right\}\)
\end{enumerate}
for alle \(t\).
Med andre ord, så afhænger \(q_{\min}, q_{\max}, u_{\max}, i_{\max}\) fra basisproblemet nu af \(t\).

\subsection{Slutvolumen og straffaktor}
\label{subsec:penalty}
Som en del af lejekontrakten i problemet kan det tænkes, at vi forpligter os til at aflevere lageret med en aftalt mængde gas, \(q_{\mathrm{goal}}\).
Hvis ikke vi rammer denne værdi præcist, bliver prisen, som modparten køber den lagrede gas for, reduceret med en \emph{straffaktor}, og vi får dermed ikke den fulde pris for gassen.
I denne udvidelse er dermed givet to nye parametre,
\begin{itemize}
\item \(\alpha\), en straffaktor for prisen (\(0 < \alpha < 1\))
\item \(q_{\mathrm{goal}}\), den aftalte mængde gas lagret til tid \(T\)
\end{itemize}
og maksimeringsproblemet bliver således
\begin{equation}
  \max_{\Delta q_{t}}
  \left\{
    - \sum_{t = 1}^{T} \e^{-r\tfrac{t}{T}} \Delta q_{t} p_{t} + \e^{-r} q_{T}p_{T}
    \left(
      1 - \alpha \cdot \1 [q_{T} \neq q_{\mathrm{goal}}]
    \right)
  \right\} ,
\end{equation}
hvor \(\1[\cdot]\) er en indikatorfunktion, dvs.\ den er lig med 1, når udsagnet (her, at \(q_{T} \neq q_{\mathrm{goal}}\)) er sandt, og 0 ellers.
Dermed er værdien af udtrykket i parentesen 1, hvis vi rammer \(q_{\mathrm{goal}}\) præcist, og \(1 - \alpha\), hvis vi ikke gør.

I denne sammenhæng kunne det være relevant at undersøge, om der findes tilfælde, hvor det kan betale sig for os \emph{ikke} at ramme \(q_{\mathrm{goal}}\), og hvilken effekt størrelsen af \(\alpha\) har.
