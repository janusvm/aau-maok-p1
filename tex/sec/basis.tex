\section{Basisproblem}
\label{sec:basis}
Forestil jer, at vi kontrollerer et gaslager---en fysisk enhed, hvori vi kan lagre naturgas---i en given periode på et år.
Med ``kontrollerer'' menes her, at vi ikke selv ejer enheden, men at vi har indgået en kontrakt med ejeren om at leje enheden.
Vi er interesserede i at maksimere profitten (eller minimere tabene), som opnås ved at købe og sælge gas i løbet af lejeperioden.

Antag hertil, at vi har en såkaldt \emph{forward}-kurve for prisen på gas, dvs.\ en række kontrakter med modparter på markedet, som giver os forudbestemte priser på gassen til hver tidspunkt i lejeperioden.
Til hvert tidspunkt i denne periode kan vi købe eller sælge en begrænset mængde gas, og der er øvre og nedre grænser på størrelsen af gaslageret (enten fysiske eller kontraktbestemte).
Når lejeperioden udløber, køber ejeren af lageret den efterladte mængde gas af os.
Vi ser desuden bort fra de omkostninger, der måtte være i en sådan lejekontrakt.

\subsection{Matematisk formulering}
\label{subsec:mathproblem}
Lad \(q_{t}\) angive mængden af gas (målt i \(\epsilon\times\)MWh, hvor \(\epsilon\) er størrelsen af en ``enhed'' gas) lagret til tid \(t\) (hvor \(t = 0\) er i dag, \(t = 1\) er næste tidsskridt osv.\ indtil slutningen af lejeperioden, \(T\)) og lad \(\Delta q_{t} = q_{t} - q_{t-1}\) angive ændringen i lageret fra den ene dag til den anden.
Givet parametrene
\begin{itemize}
\item \(q_{0}\), antal enheder af gas på lageret i starten af perioden
\item \(q_{\min}\) og \(q_{\max}\), den nedre og øvre grænse på mængden, der kan være på lageret
\item \(u_{\max}, i_{\max} \in \N\), antal enheder, der hhv.\ kan tages ud af og sættes ind på lageret mellem to tidsperioder
\item \(p_{t}\), en forward-pris i euro på en enhed gas, defineret for \(t \in \{1, \dotsc, T\}\)
\item \(r\), en fast, årlig diskonteringsrente
\end{itemize}
maksimér nutidsværdien af profitten mht.\ \(\Delta q_{t}\), dvs.\
\begin{equation}
  \label{eq:1}
  \max_{\Delta q_{t}}
  \left\{
    - \sum_{t = 1}^{T} \e^{-r\tfrac{t}{T}} \Delta q_{t} p_{t} + \e^{-r} q_{T} p_{T}
  \right\} ,
\end{equation}
under bibetingelserne
\begin{enumerate}
\item \(q_{\min} \leq q_{t} \leq q_{\max}\)
\item \(\Delta q_{t} \in
    \left\{
    -u_{\max}, -u_{\max}+1, \dotsc, -1, 0, 1, \dotsc, i_{\max} - 1, i_{\max}
    \right\}\)
\end{enumerate}
for \(t \in \{1, \dotsc, T\}\), hvor \(q_{t} = q_{0} + \sum_{s = 1}^{t} \Delta q_{s}\) som nævnt ovenfor er mængden lagret til tid \(t\).

\clearpage
\subsection{Løsning af problemet med grafteori}
\label{subsec:graphtheory}
Eftersom antallet af tidsperioder og antallet af mulige mængder gas på lageret begge er diskrete, dvs.\ \(t \in \{1, \dotsc, T\}\) og \(q_{t} \in \{q_{\min}, q_{\min}+1, \dotsc, q_{\max}\}\), kan problemet repræsenteres som en \emph{orienteret, vægtet graf}, som på skitsen nedenfor på Figur~\ref{fig:graph}:

\begin{figure}[htbp]
  \centering
  \begin{tikzpicture}[scale=1.5,-{Latex[angle'=60,length=4pt]},shorten >=1pt]
  \foreach \t [evaluate=\t as \myt using int(\t+1)] in {1, 2, ..., 5} {
    \foreach \q in {0, 1, ..., 3} {
      \node[point] at (1.25*\t,1.25*\q) (p\t\q) {};
    }
  }
  \foreach \t [evaluate=\t as \nextt using int(\t+1)] in {1, 2, ..., 4} {
    \foreach \q in {0, 1, ..., 3} {
      \foreach \nq [evaluate=\nq as \nextq using int(\q+\nq)] in {-1, 0, 1} {
        \ifthenelse{\nextq > 3 \OR \nextq < 0}{}{
          \draw (p\t\q) -- (p\nextt\nextq);
        }
      }
    }
  }
  \node[point] at (0,1.25) (q0) {};
  \foreach \q in {0, 1, 2} {
    \draw (q0) -- (p1\q);
  }
  \begin{scope}[color=red,very thick,-{Latex[angle'=65,length=6pt]},shorten >=0pt]
    \draw (q0)  -- (p12);
    \draw (p12) -- (p22);
    \draw (p22) -- (p33);
    \draw (p33) -- (p42);
    \draw (p42) -- (p51);
  \end{scope}
  \node at (q0) (lq0) [label=left:\(q_{0}\)] {};
  \node at (p10) (qmin) [label=left:\(q_{\min}\)] {};
  \node at (p13) (qmax) [label=left:\(q_{\max}\)] {};
  \node (t1) [below of=p10,yshift=1em] {\(t=1\)};
  \node (tT) [below of=p50,yshift=1em] {\(t=T\)};
  \node at (p12) (q1) [label=above:\(q_{1}\)] {};
  \node at (p22) (q2) [label=above:\(q_{2}\)] {};
  \node at (p33) (q3) [label=below:\(q_{3}\)] {};
  \node at (p42) (q4) [label=below:\(q_{4}\)] {};
  \node at (p51) (q5) [label=below:\(q_{5}\)] {};
\end{tikzpicture}

  \captionsetup{width=0.8\textwidth}
  \caption{
    \small
    Skitse af problemgrafen (kantvægte er udeladt for læsbarhed), med \(u_{\max} = i_{\max} = 1\).
    Et eksempel på en vej gennem grafen er indtegnet med rødt.
    Opadgående pile betegner indkøb af gas, mens nedadgående piler betegner salg.
  }
  \label{fig:graph}
\end{figure}

Vælges kantvægtene til at repræsentere udgifter og indtjening på hhv. køb og salg af gas, så er dette et \emph{længste vej}-problem, og det kan løses med algoritmer fra grafteori.
Hertil findes algoritmer, der kan finde den længste (eller korteste) vej igennem en graf af generel struktur, men I opfordres her til at tænke over, hvordan strukturen i dette specifikke problem kan udnyttes til at designe en mere specialiseret algoritme.
Som en del af projektet skal I således implementere en sådan algoritme i et programmeringssprog som Julia, Python eller lignende.

I genkender måske formen af dette problem som et \emph{heltalsprogrammerings}-problem (eng.: \emph{integer linear programming} (ILP)), dvs.\ et optimeringsproblem, som kan opskrives på standardformen
\begin{alignat*}{2}
 & \text{maksimér}            &\quad  & \vc^{\top} \vx  \\
 & \text{under bibetingelser} &\quad  & A \vx + \vs = \vb , \\
 &                            &\quad  & \vs \geq \vzero , \\
 &                            &\quad  & \vx \geq \vzero , \\
 & \text{og}                  &\quad  & \vx \in \Z^{n} ,
\end{alignat*}
hvor \(\vc,\vb\) og \(A\) er givne parametre, og hvor \(\vx\) og \(\vs\) er variable.
Dermed kan det ganske vist løses med software-pakker som GLPK, Excel Solver Add-in, osv., men I anbefales kun at bruge sådanne programmer til at tjekke resultaterne fra jeres egen algoritme.
En af udvidelserne i næste afsnit gør desuden objektfunktionen ikke-lineær, og så kan disse metoder ikke længere anvendes.
